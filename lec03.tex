\textit{Note:} the non-commutative algebra
$\mathbb{K}<x,y>/(yx-xy-1)$ is called the \textit{Weyl} algebra.

\section{Free presentations of modules}
Let $A$ be a ring and $M = {}_AM$ - a left $A$-module.
$A$ free presentation of $M$ is an exact sequence
\begin{equation}\label{eqn:exact-sequence}
F_1 \lra^{d_1} F_0 \lra^{\pi} M \lra 0
\end{equation}
with $F_0, F_1$ - free left $A$-modules.

Let's make sense of this definition:
Assume $F_0 = \bigoplus_{i\in I_0}{}_A^A$.
Let $\{\delta_i\}_{i\in I_0}$ be a basis for $F_0$.
Then $\{\pi(\delta_i)\}_{i\in I_0}$ generate
$M$ because $\pi$ is surjective.
$d_1$ and the elements in $F_1$ determine the 
\textit{relations} between generators $\delta_i$ of $M$.
If $x\in M$, then by surjectivity, there is 
$\sum_{i\in I_0} y_i \delta_i \in F_0$ with
$x = \sum_{i\in I_0} y_i \pi(\delta_i)$
(all but many elements in the sum are $0$).

By definition of an exact sequence, the image of $\delta_1$
is the kernel of $\pi$, so any relation 
$\sum_{i\in I_0} (a_i \pi(\delta_i)) = 0 $ in $M$
satisfies $\sum_{i\in I_0} a_i \delta_i = d_1(x)$
for some $x\in F_1$.
$x$ can be written as $x = \sum_{i\in I_1} b_i \eta_i$.
A basis $\{\eta_i\}_{i\in I_1}$
for $F_1$ gives a \textit{basis for the relations in $M$}.
The relations are called \textbf{syzygies}.

\subsection{Example of free presentation}
Let $A = \mathbb{Q}[x, y]$, $M = (x^4-y^2-1, x^2+y)$
is an $A$-module (every ideal is a module). Since there
are two generators, a presentation can begin like this:

\begin{eqnarray*}
    A \oplus A & \lra^\pi & M \lra 0\\
    (1, 0) & \mapsto & x^4-y^2-1 := f_1\\
    (0, 1) & \mapsto & x^2+y := f_2
\end{eqnarray*}

A possible relation between $f_1, f_2$ is 
$f_1f_2 - f_2f_1=0$. So $\pi(f_2, -f_1) = 0$.

Since $f_1 - (x^2-y)f_2 = -1$, we see that
$M = A = (1)$, so $(x, y)$ is a simpler 
set of generators. The only relations for $(x, y)$
is then $xy-yx=0$, which correspond
to a $1$-dimensional set $F_1$.

\section{More abstract theory}
If $A$ is $G$-filtered and ${}_AM$ is a $G$-filtered $A$-module,
we look for \textit{filtered} presentations
$F_1 \lra^{d_1} F_0 \lra^{\pi} M \lra 0$.
A \textit{filtered free presentation} is one where
the morphisms are filtered:
$d_1(F_g(F_1)) \subseteq F_g(F_0)$.

\textbf{Grading} is a functor from the category \textbf{Filt}
of $G$-filtered $A$-modules to the category
\textbf{Gr} of $G$-graded $A$ modules.

An object ($G$-filtered $A$-module) is mapped
to its associated graded object.
A filtered morphism $f: M\lra N$ is turned to
a morphism $gr(f) : gr(M)\lra gr(N)$.

The sequence
\begin{equation}\label{eqn:gr-sequence}
gr(F_1) \lra^{gr(d_1)} gr(F_0) \lra^{gr(\pi)} gr(M) \lra 0
\end{equation}
needs not (see example (\ref{ssec:example-sequence-not-good}))
be exact.

\begin{Definition}
    (\ref{eqn:exact-sequence}) is \textit{good} if 
    (\ref{eqn:gr-sequence}) is exact.
\end{Definition}

\begin{Definition}
    A generating family $(f_i){i\in I}$ of $M$ is called
    a \textit{standard basis}, if there is a filtration
    on $F_0 = \bigoplus_{i\in I}{}_AA$, such that 
    $\pi: F_0 \lra M \lra 0$ mapping $\delta_i \mapsto f_i$
    is a \textit{good filtered sequence}.
    \end{Definition}

\subsection{Example - exact sequences that are not good}
\label{ssec:example-sequence-not-good}
$k$ is any field. $A = k[x,y,z]$. The natural numbers 
$\mathbb{N}$ are graded and $A$ is given an 
$\mathbb{N}$-filtration by 
$F_n(A) = \{p \in A : \deg p \leq n\}$.
Let $M=(f_1, f_2) := (xz-1, x^3-y^3)$. 
Is $(f_1, f_2)$ a standard basis of $M$?

What kind of filtration should $F_0 = A\oplus A$ have?
We have as earlier $F_0 = A\delta_1 \oplus A\delta_2$
with $\pi(\delta_i) = f_i$. What grade should 
$\delta_1, \delta_2$ have in $F_0$?

The only way to make the graded map work is 
$\delta_1 \in F_2(F_0), \delta_1\not\in F_2(F_0)$.
Similarly $\delta_3 \in F_3(F_0) \setminus F_2(F_0)$.
This induces a filtration of $F_0$.

We have $x^2f_1 - zf_2 = y^2z - x^2 := f_3$,
which must have both grade $3$ and $4$ if 
$(f_1, f_2)$ were a \textit{standard basis}.
It turns out that 
$(f_1, f_2, f_3)$ is \todo{Why is that?} one.

$(f_1, f_2, f_3)$ - standard basis implies
\begin{eqnarray*}
A\oplus A\oplus A \lra M \lra 0\\
(a_1, a_2, a_3) \mapsto a_1 f_1 + a_2 f_2 + a_3 f_3
\end{eqnarray*}

\subsubsection{Exercise:}
Prove that $xz, x^3, y^2z$ generate $M$.

\section{Comparsion between different filtrations}
Let $G, H$ be two ordered monoids with
$\phi: G\lra H$ - a morphism between ordeded monoids.
This translates into a relation between the orders
of $G$ and $H$:
\[
    a \leq_G b \lra \phi(a) \leq_H \phi(b)
\]

\textbf{Example:} the degree lexicographic order
on $[x, y, z]$: 
$x^{a_1}y^{b_1}z^{c_1} > x^{a_2}y^{b_2}z^{c_2}$
if 
\begin{itemize}
\item $a_1 + b_1 + c_1 > a_2 + b_2 + c_2$ OR
\item $a_1 + b_1 + c_1 > a_2 + b_2 + c_2$ AND 
    $a_1 > a_2$ OR
\item $a_1 + b_1 + c_1 > a_2 + b_2 + c_2$ AND 
    $a_1 = a_2$ AND $ b_1 > b_2$.
\end{itemize}
Then  $[x,y,z] \lra \mathbb{N}$,
    $x^ay^bz^c \mapsto a+b+c$ is a morphism
    between ordered monoids.

A morphism $\phi : G\lra H$ of ordered monoids
induces an $H$-filtration on a $G$-filtered
object $A$:
\[
F_x^H(A) = \bigcup_{\phi(y) \leq x} F_y^G(A)
\]
This makes it possible to construct two graded associated
objects. 
Taking a coarser grading first and a finer
grading later, this is the same 
as taking the finer filtration directly
$gr^{(G)}\left( 
    gr^{(H)}(A)
\right) \cong gr^{(G)}(A)$ \todo{exercise! Hand in!}
