\section{Verification}

Last time, we left at showing that there exist a unique normal form.
We didn't prove that the Buchberger's algorithm terminates. It will
if the monoid is well-ordered.

\subsection{Well ordered monoids, generators and submonoids}
Let $(M, <)$ be an ordered monoid. A \textbf{system of generators} for $M$
is a set $S = \{x_i\}_{i\in I}$ of elements in $M$ such that 
any $x\in M$ is a finite product (an empty product is $1_M$)
of some $s_i$. 

For any set $S\subset M$, denote $\langle S \rangle$ be the set 
of all finite products generated by elements in $S$. This is 
an ordered submonoid of $M$, and it is the smallest submonoid of $M$
that contains $S$.

\begin{Theorem}
    \label{thm:ordered-monoid}
    If an ordered monoid $(M, <)$ is finitely generated, then it is 
    well-ordered.
\end{Theorem}

\begin{proof}
It is enough to show that $M$ satisfies the \textbf{descending chain condition}.
Assume that $x_1 > x_2 > \ldots$ is an infinite descending chain.
Then $\{x_i| i \in \mathbb{Z}_{\geq 0}\}$ has no smallest element.
Assume there exists $S\subset M$, $S\neq \emptyset$ such that $\min S$ does not exists,
then choose any $x_1 \in S$, choose any $x_2 \in S, x_2 < x_1$ and so on.
This process does not stop, since otherwise $S$ would have a minimal element.
Note that \textbf{this assumes choice}.

We showed 
\begin{eqnarray*}
    \neg\text{well order} \Lra \neg\text{d.c.c} \\
    \neg\text{d.c.c} \Lra \neg\text{well order}
\end{eqnarray*}
thus 
\[\text{well order} \Llra \text{d.c.c}\]
\end{proof}

Another equivalent claim to the \textbf{d.c.c} condition of \textbf{well-orderedness}
is that \textit{any sequnce $x_1, x_2,\ldots$ has an infinite} non-decreasing subsequence.
\begin{proof}
    We show that \textbf{well-order} implies \textbf{non-decreasing subsequence}.
    \smallLine
    Let $x_1, x_2, \ldots$ be an infinite sequence. Choose $y_1$
    as the minimal element of all $x_i$. Then choose $y_2$ as the minimal element
    of the sequence starting at the first occurence of $y_1$. Repeat.

    \myLine
    We show that \textbf{non-decreasing subsequence} implies d.c.c
    Let $x_1, \ldots, x_2$ be a descending (non-strictly) chain.
    It must have an increasing subsequence. The only way that can happen is if 
    the d.c. is eventually constant.

\end{proof}

For any well-ordered set $(A, <)$, \textbf{transfinite induction} works.
If $P$ is a property on elements of $A$ such that 
\[
x\in A \wedge (y < x \Lra P(y)) \Lra P(x)
\]
then $P$ holds for all $x\in A$.
(So if it is so that from the fact that a property holds for all $y<x$
it will follow that it holds for $x$, then it holds for all $x$).
To prove this, let $S$ be the set $\{x\in A: \neg P(x)\}$ of all elements 
that do not satisfy the property. Then $S$ does not have a smallest 
element (since it would otherwise satisfy the premise 
$x\in A \wedge (y < x \Lra P(y))$) and $S$ must be empty.




\smallLine
We now return to the proof of \textbf{Theorem \ref{thm:ordered-monoid}}.
\begin{proof}
    Let $(M, <)$ be a finitely generated monoid.
    Assume $M = \langle{x_1,\ldots, x_n}\rangle$.
    We will use induction on $n$.
    For $n=0$, the monoid is just $M = \{1_M\}$.
    Assume the theorem holds for $<n$ generators.
    Assume (WLOG) $x_1< x_2 < \ldots < x_n$.
    Let $M' := \langle\{x_1,\ldots, x_{n-1}\}$ - a well-ordered monomial
    Then let $y_1 > y_2 > \ldots $ be a descending chain on $M$.
    Then $y_1$ can be expressed as some product 
    $y_1 = x_{i_1}x_{i_2}\ldots x_{i_{r}}$.
    Let $s > r$, pick $\alpha_0, \ldots, \alpha_s \in M$ 
    and note that $\alpha_0 x_n \alpha_1 x_n \ldots x_n\alpha_s > y_1$
    for any choise of $\alpha_i$ (left as an exercise,
    hint - note that $x_{i_j} \leq x_n \alpha_i$).

    Therefore for each $y_i$ we may chose 
    an $r_i \leq r$ and $\alpha_{i, 0}, \ldots, \alpha_{i, r_i} \in M'$
    such that $y_i = \alpha_{i,0}x_n \alpha_{i, 1}x_n \ldots x_n\alpha_{i, r_i}$.
    If the sequence $y_1 > y_2 > \ldots$ were infinite, 
    then WLOG $r_i = r$ for all $i$ 

    \bigRedBox{exercise for reader, hint - 
    show that $r_i$ must eventually be constant,
    because otherwise, we would have later elements that are larger
    that the first element}.

    Consider the sequence $\alpha_{1, 0}, \alpha_{2, 0}, \ldots$.
    By the inductive hypothesis, it should contain an infinite 
    non-decreasing subsequence
    $\alpha_{i_1, 0} \leq \alpha_{i_2, 0} \ldots$.
    WLOG assume that $\alpha_{1, 0}\leq  \alpha_{2, 0}, \leq \ldots $
    is non-decreasing.
    Continue the same way and we get that the whole sequence is actually
    non-decreasing. Contradiction!    

\end{proof}

Back to our rewriting rules. 
\begin{Theorem}
If the monoid is \textbf{well-ordered}, then Buchberger's algorithm
terminates. 
\end{Theorem}

\begin{proof}
We have a finite number of them. Each rule
was of the form
\[
    m_i \mapsto t_i(x_1, \ldots, x_n)
\]
where $t_i \in k[x_1, \ldots, x_n]$ and $m_i \in [x_1, \ldots, x_n]$.
Either $t_i = 0$ or $Lm(t_i) < m_i$. Each time rule $i$ is applied
to $cm$, $c \in k^*, m\in [x_1,\ldots, x_n], m_i \mid m$,
it is reduced completely by $c\frac{m}{m_i}f_i$.

We now use \textbf{transfinite induction} over $M = [x_1, \ldots, x_n]$.
Consider the property $P$ on $M$ being 
\textbf{Any $f\in k[x_1,ildots, x_n]\setminus \{0\}$ with 
$Lm(f) = m$ has only finite reduction chains}.

Let $m\in M$ and assume $P(m')$ for all $m' < m$.
Then let $f(\ol{x}) = \sum_{i\in I}c_i q_i$, $m = q_1 < q_2 < \ldots$.
Let $g(\ol{x}) = f(\ol{x}) - c_1m$ be the \textbf{tail} of $f$.
In any reduction chain on $f(\ol{x})$, only a finite number of steps 
may be performed on $g(\ol{x})$ before $c_1m$ is considered.
Then the leading monomial is reduced (if possible) and we can use the induction
assumption, or it was not possible to reduce and we only reduce $g(\ol{x})$
(also in finitely many steps).



\end{proof}