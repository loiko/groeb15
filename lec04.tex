%\section{Questions from last time}

\section{Buchberger Algorithm}
Last time, we had $f_1 = xz-1$,
$f_2 = x^3-y^2 \in k[x,y,z]$.
We define $(f_1, f_2) := \mathfrak{a}$.

We may find $f_3=x^2f_1-zf_2=y^2z-x^2$
and show that $(f_1,f_2,f_3)$ is a standard 
basis, by means of the \textbf{Buchberger Algorithm}.
A fundamental idea is this: Each generator corresponds to
a rewriting rule. We want enough rules to ensure that each element in $\mathfrak{a}$ may be reduced to zero in steps not
involving "higher leading monomials". It will turn out that this
is equivalent to \textit{resolving all} ambiguities, and 
to have \textit{unique normal forms} for all polynomials.

\subsection{Example}

Before explaining the concepts and verifying the equivalences,
let's look at our example (with $f_1, f_2, f_2$). The leading
monic monomial of $f_1$ is $Lm(f_1) = xz$. This means that $f_1$
"represents" a generator $xz$ in $gr \mathfrak{a}$.

$gr\mathfrak{a}$ will be a monomial ideal (formally in
$gr k[x,y,z]$). This ideal will contain $xz$, (because
$xz \equiv f_1$ in $(gr \mathfrak{a})_{xz}$). 
Also $x^3z \equiv x^2f_1 \in (gr \mathfrak{a})_{x^3z}$.
Likewise, $Lm(f_2) = x^3 \in (gr \mathfrak{a})_{x^3}$.
We take $x^3z = lcm(Lm(f_1), Lm(f_2))$.
Thus $x^2f_1-zf_2=y^2z-x^2\in \mathfrak{a}$.


We add $f_3 := y^2z-x^2 = x^3z-x^2 - 
(x^3z-zy^2) = x^2f_1 - zf_2$ to our generating system, which
right now consists of $(f_1, f_2, f_3)$. We repeat the process 
using three "rules of reduction". Consider $(f_1, f_3)$:
\[
lcm(Lm(f_1), Lm(f_3)) = lcm(xz, y^2z) = xy^2z
\]
We get from this form $Lm(y^2f_1) = xy^2z = Lm(xf_3)$.
Thus, we look at $ y^2f_1 - xf_3 = y^2(xz-1) - x(f_3) = 
xy^2z - y^2 - x(y^2z-x^2) = x^2-y^2 = f_2$.
Now consider $(f_2, f_3). lcm(Lm(f_2), Lm(f_3)) = 
lcm(x^3, y^2z) = x^3y^2z$. We get
$y^2zf_2-x^3f_3 = y^2z(x^3-y^2)-x^3(y^2z-x^2)=x^5-y^4z$.
The leading monomial is divisible by $x^3$,
so we already knew it was in the graded ideal.

\subsection{...}
Let's reconsider this in terms of rewriting rules. We have
modulo $\mathfrak{a}$
the three rewriting rules $xz\mapsto 1, x^3\mapsto y^2,$
and $y^2z\mapsto x^2$. (They are "good" in the sense that 
they may be used to get polynomials with lower leading
monomials.) The following procedure defines a function
\[ \phi : k[x,y,z] \lra k[x,y,z]
\]
with the following properties
\begin{itemize}
\item $\phi$ is $k$-linear
\item $g-\phi(g)\in \mathfrak{a}$ for all $g\in k[x,y,z]$
\item $g\equiv h (\mod \mathfrak{a})$ iff $\phi(g) = \phi(h)$
\end{itemize}

This $\phi(g)$ will be called the \textit{normal form}
of $g$ (with respect to $\deg_{k[x,y,z]})$ and $\mathfrak{a}$),
and is written $Nf(g)$. $\phi$ would allow to distinguish
between cosets of $(f_1, f_2, f_2)$. The tentative definition
of $\phi$ is as follows:

\bigRedBox{
    As long as $g$ contains any monomial on which one
    of the three substitution rules may be applied,
    do apply it. The end product be $\phi(g)$.
}






